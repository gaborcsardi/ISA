\documentclass[a4paper]{article}
\usepackage{url}
\usepackage{natbib}
\usepackage{ragged2e}
\usepackage{color}

\newcommand{\Rpackage}[1]{`\texttt{#1}'}
\newcommand{\Rfunction}[1]{`\texttt{#1}'}
\newcommand{\Rclass}[1]{\textsl{#1}}

\newenvironment{myquote}{\begin{quote}\color{blue}}{\end{quote}}

\setlength{\parskip}{12pt}
\setlength{\parindent}{0pt}

\title{ExpressionView -- an interactive viewer for
  modules identified in gene expression data\\[10pt]
  Answers to reviewers}
\author{Andreas L\"uscher, G\'abor Cs\'ardi, \\ Aitana Morton de
  Lachapelle, Zolt\'an Kutalik, \\ Bastian Peter and Sven Bergmann}

\hyphenation{Bic-Over-lap-per}

\begin{document}

\maketitle

\RaggedRight

First of all, we want to thank the reviewers for their thorough work
and useful suggestions. In the following, the reviewers' comments are
typeset in blue, and indented as a block. Quotations from the
manuscript itself appear in a slanted font.

\section*{Reviewer 1}

\begin{myquote}
All comments are for minor revisions:

1) BiVisu by Cheng et al shall be cited:
Bioinformatics. 2007;23(17):2342-4. Please briefly elaborate the major
difference(s) of ExpressionView from BiVisu and BicOverlapper.
\end{myquote}

We have added a citation to Bivisu and a brief comparison to
Bivisu~\citep{cheng07} and BicOverlapper\citep{santamaria08}:

\textsl{Existing approaches include Bivisu~\citep{cheng07} and
BicOverlapper~\citep{santamaria08}. The former is an interactive
biclustering program that plots modules individually, making it
difficult to identify the relationship between the overlapping
modules. The latter is a novel tool for plotting overlapping modules,
yet in an abstract space. Our approach in ExpressionView is different,
as we use the usual gene-sample space and visualize all modules
together, on top of the reordered expression matrix. The reordering
ensures that the genes and samples that appear in the same module are
kept together.}

\begin{myquote}
2) Page 2, line 30: please re-write the sentence "We find that in
almost every situation, the proposed algorithm finds an order that
recovers more than 99\% of the score of the optimal solution and in
most cases, it recovers the correct alignment". 
\end{myquote}

We have cut the ambiguous part of the sentence, it now reads:

\textsl{We find that the proposed algorithm finds an
order that recovers more than 99\% of the score of the optimal
solution and in most cases, it recovers the correct
alignment.}


\begin{myquote}
3) Page 2, line 55, change "any computer" to "a computer".
\end{myquote}

We have corrected this.

\section*{Reviewer 2}

\begin{myquote}
Comments to the Author

This is a very useful tool which users might  benefit if there is
clearly stated in the website, what sort of input format the
application accepts. There does not seem to be a complete and explicit
description of the various types of input format beyond just a
expression matrix.
\end{myquote}

For the matrix reordering, ExpressionView accepts modules in the
format defined in the eisa R package~\citep{csardi10}
(\textit{ISAModules} object) and the
Biclust R package~\citep{kaiser08} (\textit{Biclust} object). These
cover the existing R implementations of biclustering algorithms (to our
knowledge). Moreover, the Biclust package defines a very simple class
for storing biclustering results, so the output of any other algorithm
(implemented in R or not) can be easily transformed into this
format. This is documented in the Biclust package.

For the expression data itself, ExpressionView accepts
\textit{ExpressionSet} objects as the input. This is the standard data
type in BioConductor for storing expression data. E.g. the GCRMA and
other normalization methods all output this format and it is also easy
to create an \textit{ExpressionSet} object from a simple tabular text
file.

Alternatively, ExpressionView can visualize non-expression data,
provided as a simple matrix, this is explained in the R manual that
comes with the package, including examples. This manual is also
available on the ExpressionView website now, under ``Tutorials''.

For the Flash application itself, we have extensively documented the
ExpressionView XML file format. This is documentation is available on
the website, under ``Tutorials''.

\begin{myquote}
For example, how do we incorporate GO terms and
KEGG pathways into the raw input dataset, or are these also computed
by the application? 
\end{myquote}

These are computed by the R part of the application, using functions
of the eisa R package. There are examples for this in the manual that
comes with the package. This manual is also available now on the
ExpressionView website.

\begin{myquote}
It is not entirely clear to the casual reader,
whether the Flash application takes raw data or the intermediate XML
result from the workflow, which means user may need to use command
line R anyway to generate the data. If it does so, the format of this
needs to be clearly stated for a non-bioinformatics biologist to be
able to load and run the application. 
\end{myquote}

The Flash application takes the intermediate XML file, as shown
in Fig.~1. The idea is that the user performing the analysis in
R is usually not the same as the user browsing the results. So the
(bioinformatician) user generates the data in R, exports it into an
XML file, and the other (e.g. life scientist) user explores it using an
ordinary web browser. We have made this distinction clear in the
manuscript:

{\slshape
With the ExpressionView package, bicluster analysis can be separated
into two parts. The first part involves finding the
modules in the data set with some algorithm, possibly running
enrichment analysis for the modules, and reordering the rows and
columns of the expression matrix according to the modules. This
part is typically done by a bioinformatician. The second part of the
analysis involves the visualization and interactive exploration of the
results. This part is typically done by researchers without extensive
programming knowledge.

The first part of is written in GNU
R~\citep{R} and contains an implementation of the matrix reordering
algorithm. The second part is an interactive visualization tool in
the form of an Adobe Flash applet, for which the user only needs a
Flash-enabled web browser.

This dual implementation has the advantage that all the power of the
GNU R environment and the BioConductor~\citep{gentleman04} packages
can be used for the analysis itself, e.g. all organisms that are (and
will be) supported by BioConductor are automatically supported by
ExpressionView. On the other hand, the exploration of the results does
not need any GNU R knowledge and in most cases no extra software needs to
be installed. See Fig.~1 for a typical
ExpressionView workflow.
}

The file format of the XML file is not relevant in most cases, as the
file is produced by the ExpressionView package, and the user does not
have to produce it by hand. It is only relevant, if the user wants to
import data that was produced by another reordering algorithm, into
the Flash applet. We have created an XML Schema document that precisely
defines the XML data file's format, and also written a document that
discusses the format. These are available on the ExpressionView website.

\begin{myquote}
The tutorials are impressive, but the audio volume could be increased
or spoken by a person with a higher pitch voice. Very difficult to
hear.  
\end{myquote}

We have increased the audio volume in the video tutorials.

\begin{myquote}
ExpressionView source code (for the Flash applet) in winrar
format contains a lot of empty folders and tmp files some of which may
not be necessary. 
\end{myquote}

We have simplified the directory structure of the source code and 
removed all empty folders.

\begin{myquote}
The paper could do with at least a few words in the body of the paper
or in the website, to explain what improvement over features of other
existing Expression data biclustering visualisers currently
available. 
\end{myquote}

We have included a brief comparison to Bivisu~\citep{cheng07} and
BicOverlapper~\citep{santamaria08}, two existing bicluster
visualization software packages:

\textsl{Existing approaches include Bivisu~\citep{cheng07} and
BicOverlapper~\citep{santamaria08}. The former is an interactive
biclustering program that plots modules individually, making it
difficult to identify the relationship between the overlapping
modules. The latter is a novel tool for plotting overlapping modules,
yet in an abstract space. Our approach in ExpressionView is different,
as we use the usual gene-sample space and visualize all modules
together, on top of the reordered expression matrix. The reordering
ensures that the genes and samples that appear in the same module are
kept together.
}

\begin{myquote}
Moreover, the paper must a least emphasize justification
for why biologists should find this application/algorithm more useful
over other existing applications/algorithms besides just visual
appeal. 
\end{myquote}

We emphasized the advantages of ExpressionView more in the paper:
\begin{itemize}
\item visualization of all the modules and the expression values
  together:

  \textsl{\noindent
    Our approach in ExpressionView is different,
    as we use the usual gene-sample space and visualize all modules
    together, on top of the reordered expression matrix. The reordering
    ensures that the genes and samples that appear in the same module are
    kept together.
  }
\item implementation as a BioConductor package:

  \textsl{\noindent
    This dual implementation has the advantage that all the power of the
    GNU R environment and the BioConductor~\citep{gentleman04} packages
    can be used for the analysis itself, e.g. all organisms that are (and
    will be) supported by BioConductor are automatically supported by
    ExpressionView. 
  }
\item ease of use for the (life scientist) end user, only a web browser is
  needed, no extra programs need to be installed.

  \textsl{\noindent
    On the other hand, the exploration of the results does
    not need any GNU R knowledge and in most cases no extra software needs to
    be installed.    
  }
\item the inclusion of meta-data and enrichment data into the same
  environment, with interactive links to corresponding databases:

  \textsl{\noindent
    On the right-hand side, the metadata
    associated with the expression data and the results of the
    enrichment calculations for GO~\citep{ashburner00} categories and
    KEGG~\citep{kanehisa04} pathways are shown. Wherever possible, these elements
    are linked to the corresponding databases.    
  }
\end{itemize}

\section*{Reviewer 3}

\begin{myquote}
Comments to the Author

General comments:

Luescher et al. describe software for visualizing modules in gene
expression data using an efficient implementation of matrix
reordering. Such a package is of high practical value for biologists
to analyze and visualize gene expression data, and I therefore
recommend publication as an Applications Note. In particular, I
appreciate the connection to the Bioconductor package, it is well
done.

Specific comments:

Minor:

I would recommend a different name for the package, ExpressionView is
too general.
\end{myquote}

While this is a just criticism, we cannot easily change the name of the
package, as it is part of BioConductor, and the name change would
cause problems for the users and the BioConductor developers.

\begin{myquote}
Biologists should have no hurdles to access and apply the software
\ldots I could not easily install it on my LINUX/Ubuntu computer after
having installed the isa2 package, as recommended by the authors. 
I assume the download and installation of the package will be as
straightforward as the isa2 package by the authors.
\end{myquote}

The ExpressionView package is an official BioConductor package since
April 2010, so its installation should be easy on any platform
supported by BioConductor (Linux, MS Windows, Apple OSX). We have
updated the installation instructions on the ExpressionView homepage
accordingly.

\begin{myquote}
The authors should consider that some computers might not play the
Flash applets, and should provide short and clear instructions for
installation on their website.
\end{myquote}

We have added a note about this, and a link to the relevant Adobe
website, to the ExpressionView homepage.

\bibliographystyle{natbib}
\bibliography{expressionview}

\end{document}

